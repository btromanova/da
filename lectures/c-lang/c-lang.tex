%\documentclass[11pt,a4paper]{article}
%\usepackage{fullpage}
%\usepackage{beamerarticle}
%\documentclass[handout,xcolor=pdftex,dvipsnames,table]{beamer}
\documentclass[hyperref={unicode=true}]{beamer}

%\usepackage{pgfpages} 
%\pgfpagesuselayout{resize}[a4paper,border shrink=5mm,landscape] 

\usepackage[utf8]{inputenc}
\usepackage[russian]{babel}
\usepackage{../clrscode3e} 
%\usepackage[all]{xy}
\usepackage{colortbl}
%\usepackage{xcolor}
%\usepackage{pstricks, pst-tree, pst-node}
\usepackage{epsfig}
\usepackage{multicol}
\usepackage{listings}

\definecolor{orange}{cmyk}{0,0.52,1,0}

%\usepackage{beamerthemesplit}

\AtBeginSubsection[]
{
  \begin{frame}<beamer>{Раздел}
    \tableofcontents[currentsection,currentsubsection]
  \end{frame}
}

\newtheorem{rtheorem}{Теорема} 
%default}
%themesplit}

\title{Программирование на языке C}
\subtitle{Дискретный анализ 2012/13}
\author{Андрей Калинин, Татьяна Романова}
\date{3 сентября 2012 г.}
\usetheme{default}
%\usefonttheme{serif}
\usefonttheme[onlymath]{serif}
%\usefonttheme{professionalfonts}
%\usetheme{default} 


\begin{document}

\frame{\titlepage}

%\section[Содержание]{}
\frame{\tableofcontents}

%\section{Литература}
\frame
{
  \frametitle{Литература}

  \begin{itemize}
  \item Кернинган, Ричи: Язык программирования C.
  \end{itemize}
}
\lstset{language=C}

\section{Библиотеки}

\subsection{libc} 

\begin{frame}[fragile]
\frametitle{Скомпилируется ли эта программа?}
\begin{lstlisting}
main() {
  printf("Hello, world!\n");
}
\end{lstlisting} \pause
\begin{verbatim}
$ gcc t01.c 
t01.c: In function 'main':
t01.c:2: warning: incompatible implicit declaration of 
built-in function 'printf'
$ ./a.out
Hello, world!
\end{verbatim}
\end{frame}

\begin{frame}[fragile]
\frametitle{Убираем предупреждение}
\begin{lstlisting}
int printf(const char *, ...);

main() {
  printf("Hello, world!\n");
}
\end{lstlisting} 
\begin{verbatim}
$ gcc t02.c 
$ ./a.out
Hello, world!
\end{verbatim}
%Почему достаточно этой строчки?
\end{frame}

\begin{frame}[fragile]
\frametitle{Что содержится в stdio.h?}
\begin{lstlisting}
#ifndef _STDIO_H_
#define _STDIO_H_
 ...
typedef struct __sFILE {
  unsigned char *_p; /*current position in buffer*/
  int     _r;    /* read space left for getc() */
  int     _w;    /* write space left for putc() */
 ...
}
 ...
int      printf(const char * __restrict, ...);
 ...
#endif /* _STDIO_H_ */
\end{lstlisting}
\end{frame}

\begin{frame}
\frametitle{stdio.h}
\begin{enumerate}
\item Определяются структуры, константы, макросы. 
\item Нет определения функций, только объявления. 
\item Включаются другие заголовочные файлы. 
\item Защищающие директивы препроцессора (ifndef/define в начале,
  endif в конце файла) 
\end{enumerate}

Где находится определение функции printf? \pause В libc.a (обычно находящейся
в каталоге /lib, /usr/lib или /usr/local/lib)
\end{frame}

\begin{frame}
\frametitle{Что такое libc?}
\begin{enumerate}
\item Библиотека стандартных функций языка С и основного интерфейса
  операционной системы Unix.
\item Подключается статически (libc.a) или динамически (libc.so).
\item libc.a: архив объектных файлов, созданный программой ar. Можно
  посмотреть содержимое командой ar t libc.a 
\item Неявно подключается при создании выполнимой программы
  компилятором gcc.
\item libc.so: более сложная структура, похожая на выполнимый
  файл, реально подключается на этапе выполнения программы. 
\end{enumerate}
\end{frame}

\begin{frame}[fragile]
\frametitle{Содержимое libc.a}
\begin{verbatim}
$ ar t /usr/lib/libc.a 
wmemset.o
wmemmove.o
wmemcmp.o
...
printf.o
...
\end{verbatim}

%printf.o: объектный файл, содержащий в себе определение функции printf.
\end{frame}

\begin{frame}[fragile]
\frametitle{/usr/src/lib/libc/stdio/printf.c}
\begin{lstlisting}
#include <sys/cdefs.h>
#include <stdio.h>
#include <stdarg.h>

int printf(char const * __restrict fmt, ...)
{
        int ret;
        va_list ap;

        va_start(ap, fmt);
        ret = vfprintf(stdout, fmt, ap);
        va_end(ap);
        return (ret);
}

\end{lstlisting}
\end{frame}

\subsection{Процесс компиляции}

\begin{frame}
\frametitle{Основные этапы сборки исполняемого файла}
\begin{enumerate}
\item Обработка препроцессором, cpp: выполнение \#-директив
  \texttt{\#include}, \texttt{\#define}, \texttt{\#ifdef} и т.п.
\item Компиляция (выполняет и препроцессор): gcc -c file.c, получается
  объектный файл. 
\item Редактирование связей: ld, объединяет все объектные
  файлы, как указанные явно, так и содержащиеся в библиотеках, в
  исполняемый файл. 
\end{enumerate}
\end{frame}

\begin{frame}
\frametitle{Директивы препроцессора}
\begin{itemize}
\item Выполняются для всех компилируемых файлов. 
\item Нужно следить, чтобы подключались одни и те же файлы: одинаковые
  настройки при компиляции, include-пути и прочее.
\item Препроцессор --- не компилятор! Множество подводных камней. 
\item Тем не менее, широко используется. К примеру,
  /usr/include/sys/queue.h, реализация очереди целиком на
  препроцессере. 
\item Популярно комментирование блоков при помощи препроцессора:
  \texttt{\#if 0}  ... \texttt{\#endif}.
\item Препроцессор может замедлять выполнение компиляции: pre-compiled
  заголовочные файлы.
\end{itemize}
\end{frame}

%\begin{frame}
%\frametitle{Вызов подпрограммы}
%\begin{enumerate}
%\item В стек помещается адрес возврата, куда должно вернуться
%  управление из подпрограммы после её окончания. 
%\item В стек помещаются аргументы (C-соглашение: сначала последний, в
%  конце первый)
%\item В стеке размещаются внутренние переменные подпрограммы. 
%\item При выходе из подпрограммы стек <<раскручивается>> в обратную
%  сторону. 
%\end{enumerate}
%\end{frame}

\begin{frame}
\frametitle{Компиляция}
\begin{itemize}
\item Компилятору нужно только описание вызова подпрограммы. 
\item Если такого нет, он сгенерирует описание неявно: int foo(...)
\item При передаче указателей на структуры не нужно описание самих
  структур. 
%\item C- соглашения о вызове, возожность указания
%  неограниченного количества аргументов. 
\end{itemize}
\end{frame}

\begin{frame}
\frametitle{Редактор связей}
\begin{itemize}
\item Производит поиск символьных названий и заменяет их на адреса
  настоящих переменных и функций. 
\item Общий для C, C++ и множества других языков: ld. 
\item Имеет собственные ограничения (например, на максимальное
  количество символов в названии функции или символ подчёркивания). 
\item В некоторых случаях важен порядок указания объектных файлов и
  библиотек. 
\end{itemize}
\end{frame}

\begin{frame}
\frametitle{Динамические библиотеки}
\begin{itemize}
\item Используются неявно (можно отключить при помощи ключа -static)
\item Реальное подключение происходит во время выполнения.
\item DLL HELL: разные версии одной и той же библиотеки на
  разных компьютерах.
\item Проверка версионности через названия файлов, /lib/libc.so.5.
\end{itemize}
\end{frame}


\subsection{Традиции языка C}

\begin{frame}
\frametitle{Модульное программирование}
\begin{itemize}
\item Отсутствует в языке, но есть ряд соглашений по совместному использованию
  препроцессора и компоновщика.
\item Интерфейс располагается в заголовочных h-файлах.
\item Реализация скрыта в одном c-файлах. 
\item Для использования функций модуля в нескольких конфигурациях
  используются хендлы (объекты). 
\item ООП лишь более явная реализация идеи модульного
  программирования. 
\end{itemize}
\end{frame}

\begin{frame}
\frametitle{Примеры объектов в языке C}
\begin{itemize}
\item Файлы UNIX: хендлы целые числа; представляют из себя сетевые
  соединения, устройства, файловую систему и т.п.
\item stdio: FILE* --- объект; fopen/popen/fcreate --- конструкторы;
  fclose --- деструктор; fprintf/fwrite/fread --- методы. 
\end{itemize}
\end{frame}

\begin{frame}[fragile]
\frametitle{Интерфейс словаря}
\begin{lstlisting}
#ifndef __DICT_H__
#define __DICT_H__

#ifdef __cplusplus
extern "C" {
#endif
  /* ... */
#ifdef __cplusplus
}
#endif

#endif /* __DICT_H__ */
\end{lstlisting}
\end{frame}

\begin{frame}[fragile]
\frametitle{Интерфейс словаря}
\begin{lstlisting}
typedef struct dict dict;

dict *dict_create();
dict *dict_load(const char *fname);
int   dict_add(dict *d, const char *key, int val);
int   dict_find(dict *d, const char *key, int *val);
int   dict_save(dict *d, const char *fname);
int   dict_close(dict *d);

\end{lstlisting}
\end{frame}

\begin{frame}[fragile]
\frametitle{Реализация словаря}
\begin{lstlisting}
#include "dict.h"

struct dict {
  /* ... */
}

dict *dict_create() {
  dict *result = calloc(1, sizeof(dict));
  /* ... */
  return result;
}
\end{lstlisting}
\end{frame}

\begin{frame}
\frametitle{Что не является раздельной компиляцией}
Вынесение всех функций в заголовочный файл не позволит сделать
библиотеку, т.к. ими можно будет пользоваться только в одной единице
компиляции. 
\end{frame}

\section{Динамическая память}
  \begin{frame}<beamer>{Раздел}
    \tableofcontents[currentsection,currentsubsection]
  \end{frame}
%\subsection{Работа с памятью в языке C}

\begin{frame}
\frametitle{Какая бывает память?}
\begin{itemize}
\item Сегмент данных: статические переменные, выделяется на этапе компиляции. 
\item Стек: аргументы функций, адреса возврата, локальные переменные
  функций, выделяется на этапе времени выполнения. Время жизни определяется порядком
  выполнения функций. Доступен через alloca.
\item Куча (динамическая память). Выделяется и уничтожается
  самостоятельно программистом. 
\end{itemize}
\end{frame}

\begin{frame}
\frametitle{Работа с кучей}
\begin{itemize}
\item Попросить память: malloc. 
\item Изменить размер блока: realloc.
\item Вернуть память: free. 
\end{itemize}
\end{frame}

\begin{frame}
\frametitle{Куча}
\begin{itemize}
\item Куча может быть системной или библиотечной. 
\item Куча фрагментируется от частого получнеия и возвращения
  памяти.
\item Библиотечная куча может не возвращать память системе. 
\end{itemize}
\end{frame}

%\begin{frame}
%\frametitle{Альтернативные способы работы с памятью}
%\begin{itemize}
%\item mmap.
%\item Разделяемая память. 
%\end{itemize}
%\end{frame}

\section{Профилирование}
  \begin{frame}<beamer>{Раздел}
    \tableofcontents[currentsection,currentsubsection]
  \end{frame}

%\subsection{Общие замечания}

\begin{frame}
\frametitle{Профайлеры}
\begin{itemize}
\item Наблюдение изнутри (gprof, gcov) и снаружи (shark) программы.
\item Изнутри: теоретически точнее, но может влиять на ход выполнения. 
\item Снаружи: практически значимый результат, но иногда не хватает
  точности. 
\end{itemize}
\end{frame}

%\subsection{gprof}

\begin{frame}
\frametitle{gprof}
\begin{itemize}
\item gcc -pg
\item Запустить программу. 
\item gprof имя-программы имя-проргаммы.gmon
\end{itemize}
\end{frame}

%\subsection{gcov}

\begin{frame}
\frametitle{gcov}
\begin{itemize}
\item gcc -ftest-coverage -lgcov
\item Запустить программу. 
\item gcov путь-к-исходнику
\end{itemize}
\end{frame}

%\subsection{dmalloc}


\section{C++}
\subsection{Особенности}
\begin{frame}
\frametitle{ld ничего не знает о С++}
\begin{itemize}
\item Шаблоны?
\item Члены классов?
\item Функции с одинаковыми названиями, но разными аргументами?
\item Операторы?
\end{itemize}

Как это всё увязать с C-шным редактором связей?
\end{frame}

\begin{frame}
\frametitle{Изменение имён функций}
\begin{itemize}
\item C добавляет символ подчёркивания. 
\item C++ добавляет всю информацию о вызове функции в название. Для gcc:
\begin{itemize}
  \item int A::f(): \_ZN1A1fEv
  \item int A::f(int): \_ZN1A1fEi
  \item int A::f(int, int): \_ZN1A1fEii
  \item int A::f(double): \_ZN1A1fEd
\end{itemize}
\item c++filt.
\item export "C"
\end{itemize}
\end{frame}

\subsection{Недостатки}

\begin{frame}
\frametitle{Проблемы}
\begin{itemize}
\item Сложная грамматика: \texttt{AA BB(CC);} --- объект или функция?
\item STL: сложнее исходной задачи, при этом нет многих необходимых .
\item Исключения: как освобождать ресурсы?
\item Перегрузка операторов (operator<< для вывода, operator/ для
  конкатенации строк)
\item Шаблоны и inline-функции: всё в заголовчных файлах. 
\item Сложный стандарт: нет ни одного компилятора, ему
  соответствующего. 
\item Каждый C++-компилятор имеет свои ошибки и особенности. 
\item Диалекты языка промышленных компиляторов: Microsoft Visual C++,
  gcc. 
\end{itemize}
\end{frame}

\begin{frame}[fragile]
\frametitle{Сообщения об ошибках}
\begin{lstlisting}
typedef std::map<std::string,std::string> 
  StringToStringMap;
void print(const StringToStringMap& dict) {
  for(StringToStringMap::iterator p=dict.begin(); 
      p!=dict.end(); ++p) {
    std::cout << p->first << " -> " 
        << p->second << std::endl;
  }
}
\end{lstlisting}
\end{frame}

\begin{frame}[fragile]
\frametitle{Текст ошибки}
\begin{verbatim}
test.cpp: In function 'void print(const StringToStringMap&)':
test.cpp:8: error: conversion from
'std::_Rb_tree_const_iterator<std::pair<const 
std::basic_string<char, std::char_traits<char>, 
std::allocator<char> >, std::basic_string<char,
std::char_traits<char>, std::allocator<char> > > >' 
to non-scalar type 'std::_Rb_tree_iterator<
std::pair<const std::basic_string<char,
std::char_traits<char>, std::allocator<char> >, 
std::basic_string<char, std::char_traits<char>, 
std::allocator<char> > > >' requested
\end{verbatim}
\end{frame}

\end{document}
    
