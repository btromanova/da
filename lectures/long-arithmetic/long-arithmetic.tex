%\documentclass[11pt,a4paper]{article}
%\usepackage{fullpage}
%\usepackage{beamerarticle}
%\documentclass[handout,xcolor=pdftex,dvipsnames,table]{beamer}
\documentclass[hyperref={unicode=true}]{beamer}

%
%
% Нужно ещё:
%  - возведение в степень. 
%  - формальности в начале. 
%  - пояснить соответствие примера на деление с фактом v_n<b/2 при q>=q^q+3
%  - классических алгоритмов с формальностями хватает на одну лекцию. 
%

%\usepackage{pgfpages} 
%\pgfpagesuselayout{resize}[a4paper,border shrink=5mm,landscape] 

\usepackage[utf8]{inputenc}
\usepackage[russian]{babel}
\usepackage{../clrscode3e} 
%\usepackage[all]{xy}
\usepackage{colortbl}
%\usepackage{xcolor}
\usepackage{pstricks, pst-tree, pst-node}
\usepackage{epsfig}
\usepackage{multicol}
\usepackage{array}
%\usepackage{listings}

\definecolor{orange}{cmyk}{0,0.52,1,0}

% \usepackage{beamerthemesplit}

\AtBeginSubsection[]
{
  \begin{frame}<beamer>{Раздел}
    \tableofcontents[currentsection,currentsubsection]
  \end{frame}
}


\newtheorem{rtheorem}{Теорема} 
\newtheorem{rconsequence}{Следствие} 
%default}
%themesplit}

\title{Длинная арифметика}
\subtitle{Дискретный анализ 2012/13}
\author{Андрей Калинин, Татьяна Романова}
\date{16 февраля 2013\,г.}
\usetheme{default}
%\usefonttheme{serif}
\usefonttheme[onlymath]{serif}
%\usefonttheme{professionalfonts}
%\usetheme{default} 


\begin{document}

\frame{\titlepage}

%\section[Содержание]{}
\frame{\tableofcontents}

%\section{Литература}
\frame
{
  \frametitle{Литература}


  \begin{itemize}
  \item Дональд Кнут, <<Искусство программирования>>, том 2,
    <<Получисленные алгоритмы>>, 3-е издание. Глава 4{.}3,
    <<Арифметика многкратной точности>>, стр. 304--335. 
  \end{itemize}
}

\section{Классические алгоритмы}

\frame{
\frametitle{База}
\begin{itemize}
  \item Целые числа представляются в системе счисления по основанию $b$, не
    превышающему машинное слово. 
  \item В наличии: 
    \begin{enumerate}
      \item Сложение и вычитание одноразрядных целых чисел с
        получением одноразрядного результата и разряда переноса. 
      \item Перемножение двух одноразрядных чисел с получением
        двухразрядного результата. 
      \item Деление двухразрядного целого числа на одноразрядное,
        дающее одноразрядное частное и одноразрядный остаток. 
    \end{enumerate}
  \item Считаем числа неотрицательными. 
\end{itemize}
}

\subsection{Сложение}
\frame{
  \frametitle{Задача}
  Дано:
  \begin{itemize}
    \item  $u = (u_{n-1}\ldots u_1u_0)_b$
    \item  $v = (v_{n-1}\ldots v_1v_0)_b$
  \end{itemize}
  Нужно найти число 
\[ 
w = u+v 
\] в представлении 
\[
w = (w_nw_{n-1}\ldots w_1w_0)_b.
\] 
Здесь $w_n$ --- разряд переноса,
  равный 0 или 1. 
}

\frame{
\frametitle{Алгоритм сложения}
\begin{codebox}
\li $k \gets 0$
\li \For $j \gets 0$ \To $n-1$ \Do 
\li $w_j \gets (u_j+v_j+k)\mod b$
\li $k \gets \lfloor (u_j+v_j+k)/b \rfloor$ \End
\li $w_n \gets k$
\end{codebox}
}

\subsection{Вычитание}

\frame{
  \frametitle{Задача}
  Дано:
  \begin{itemize}
    \item  $u = (u_{n-1}\ldots u_1u_0)_b$
    \item  $v = (v_{n-1}\ldots v_1v_0)_b$
    \item $u \geq v$
  \end{itemize}
  Нужно найти число 
\[ 
w = u-v 
\] в представлении 
\[
w = (w_{n-1}\ldots w_1w_0)_b.
\] 
}

\frame{
\frametitle{Алгоритм вычитания}
\begin{codebox}
\li $k \gets 0$
\li \For $j \gets 0$ \To $n-1$ \Do 
\li $w_j \gets (u_j-v_j+k)\mod b$
\li $k \gets \lfloor (u_j-v_j+k)/b \rfloor$ \End
\end{codebox}
В конце алгоритма $k = 0$, по начальному предположению $u \geq v$.
}


\subsection{Умножение}
\frame{
  \frametitle{Задача}
  Дано:
  \begin{itemize}
    \item  $u = (u_{m-1}\ldots u_1u_0)_b$
    \item  $v = (v_{n-1}\ldots v_1v_0)_b$
  \end{itemize}
  Нужно найти число 
\[ 
w = u \times v 
\] в представлении 
\[
w = (w_{m+n-1}\ldots w_1w_0)_b.
\] 
}

\frame{
\frametitle{Алгоритм умножения}
\begin{codebox}
\li $(w_{m-1}, w_{m-2}, \ldots , w_0) \gets (0, 0, \ldots, 0)$
\li \For $j \gets 0$ \To $n-1$ \Do
\li \If $v_j \neq 0$ \Then
\li $k \gets 0$
\li \For $i \gets 0$ \To $m-1$ \Do
\li $t \gets u_i \times v_j + w_{i+j} + k$
\li $w_{i+j} \gets t \mod b$
\li $k \gets \lfloor t/b \rfloor$ \End
\li $w_{j+m} \gets k$  \End \End
\end{codebox}

Алгоритм работает за $O(nm)$, есть значительно более быстрые
алгоритмы. 
}

\subsection{Деление}

\frame{
  \frametitle{Пример деления столбиком}

\begin{columns}
\begin{column}{.5\textwidth}
\begin{tabular}{ll}
\texttt{1260257} & \multicolumn{1}{|l}{\texttt{37}} \\
\cline{2-2}
\only<2->{\texttt{111}} & \multicolumn{1}{|l}{\texttt{\only<2->3\only<4->4\only<5->0\only<7->6\only<9->1}} \\
\texttt{\only<2->{~15}\only<3->0}\\
\only<4->{\texttt{~148}}\\
\texttt{~~~\only<4->2\only<5->2\only<6->5}\\
\texttt{~~~\only<7->{222}}\\
\texttt{~~~~~\only<7->3\only<8->7}\\
\texttt{~~~~~\only<9->{37}}\\
\texttt{~~~~~~\only<9->0}\\
\end{tabular}
\end{column}
\begin{column}{.5\textwidth}
\only<10->{Самое сложноформализуемое  в этом алгоритме --- определение следующего
  разряда частного.}
\end{column}
\end{columns}
}

\frame{
  \frametitle{Задача получения следующего разряда}
  Дано:
  \begin{itemize}
    \item  $u = (u_{n}u_{n-1}\ldots u_1u_0)_b$
    \item  $v = (v_{n-1}\ldots v_1v_0)_b$
    \item $u/v < b$ (или $(u_n u_{n-1}\ldots u_0)_b <
      (v_{n-1}v_{n-2}\ldots v_0)_b$)
  \end{itemize}
  Нужно найти число 
\[ 
q = \lfloor u/v \rfloor 
\] 
}

\frame{
  \frametitle{Угадывание $q$}
  Положим 
\[
\hat{q} = \min \left( \left\lfloor \frac{u_n b + u_{n-1}}{v_{n-1}}\right\rfloor ,
  b-1 \right)
\]
\pause
Как это ни странно, такое приближение даёт хорошие результаты. Покажем
это. 
}
\frame{
\begin{rtheorem}
$\hat{q}=\min \left( \left\lfloor \displaystyle\frac{u_n b + u_{n-1}}{v_{n-1}}\right\rfloor ,
  b-1 \right)\geq q$
\end{rtheorem}
  \begin{proof}
    $\hat{q} = b-1\quad\Rightarrow\quad \hat{q} \geq
    q$. 
\pause

$\hat{q} = \lfloor (u_n b + u_{n-1})/v_{n-1}\rfloor \quad \Rightarrow
\quad \hat{q}v_{n-1} \geq u_nb + u_{n-1}-v_{n-1}+1$.
\pause

\begin{align*}
u - \hat{q}v &\leq u - \hat{q}v_{n-1}b^{n-1} \\
&\leq u_nb^n + \cdots + u_0 - (u_n b^n +
u_{n-1}b^{n-1}-v_{n-1}b^{n-1}+b^{n-1}) \\
&= u_{n-2}b^{n-2} + \cdots + u_0 - b^{n-1}+v_{n-1}b^{n-1} \\
& < v_{n-1}b^{n-1}\leq v
\end{align*}
\pause
$u  - \hat{q}v < v \quad \Rightarrow \quad \hat{q}\geq q$ (т.к. $0\leq
u-qv<v$).
  \end{proof}
}

\frame[plain]{
  \frametitle{Насколько $\hat{q}$ больше $q$?}
  Преположим, что $\hat{q}\geq q+3$. Тогда:
\[
\hat{q} \leq \frac{u_nb+u_{n-1}}{v_{n-1}} =
\frac{u_nb^n+u_{n-1}b^{n-1}}{v_{n-1}b^{n-1}} \leq \frac
{u}{v_{n-1}b^{n-1}} < \frac{u}{v-b^{n-1}}
\]
\pause
Т.к. $q >(u/v)-1$, то:
\[
3 \leq \hat{q}-q < \frac{u}{v-b^{n-1}}-\frac{u}{v}+1 = \frac{u}{v}\left(\frac{b^{n-1}}{v-b^{n-1}}\right)+1
\]
\pause
Следовательно:
\[
\frac{u}{v}> 2\left(\frac{v-b^{n-1}}{b^{n-1}}\right) \geq 2(v_{n-1}-1)
\]
\pause
Поскльку $b-4\geq \hat{q}-3\geq q = \lfloor u/v \rfloor \geq
2(v_{n-1}-1)$, то $v_{n-1}<\lfloor b/2 \rfloor$.
}

\frame{
  \begin{rtheorem}
    Если $v_{n-1} \geq \lfloor b/2 \rfloor$, то $\hat{q}-2\leq q \leq
    \hat{q}$. 
  \end{rtheorem}
\pause
  Следствия: 
  \begin{enumerate}
    \item Условие теоремы не зависит от того, насколько велико $b$: подобранное
      частное $\hat{q}$ не отличается от истинного более чем на 2. 
\pause
    \item Любые входные данные можно преобразовать таким образом,
      чтобы выполнялось условие теоремы. Например, можно домножить $u$
      и $v$ на $\lfloor b/(v_{n-1}+1)\rfloor$. Это не изменит значения
      $u/v$ и не увеличит количество разрядов в $v$, при этом новое
      $v_{n-1}$ станет достаточно велико. 
  \end{enumerate}
}

\frame{
  \frametitle{Дополнительные условия}
  $\hat{r} = u_nb +u_{n-1}-\hat{q}v_{n-1}$. Предположим, что
    $v_{n-1}>0$.
  \begin{enumerate}
  \item Если $\hat{q}v_{n-2}>b\hat{r}+u_{n-2}$, то
    $q<\hat{q}$.
  \item Если $\hat{q}v_{n-2}\leq b\hat{r}+u_{n-2}$, то  $\hat{q}=q$
    или $q=\hat{q}-1$.
  \item Если $v_{n-1}\geq \lfloor b/2 \rfloor$ и $\hat{q}v_{n-2} \leq
    b\hat{r}+u_{n-2}$, но $\hat{q}\neq q$, то $u \mod v \geq
    (1-2/b)v$, т.е. это событие происходит с вероятностью приблизительно
    равной $2/b$.
  \end{enumerate}
}

\frame{
  \frametitle{Задача длинного деления}
  Дано:
  \begin{itemize}
    \item  $u = (u_{n+m-1}u_{n-1}\ldots u_1u_0)_b$
    \item  $v = (v_{n-1}\ldots v_1v_0)_b$
    \item $v_{n-1} \neq 0$, $n>1$.
  \end{itemize}
  Нужно найти числа $q$ и $r$, такие, что:
\[
u = v\times q + r,\quad 0 \leq r < v
\]
в представлении 
\[ 
q = (q_mq_{m-1}\ldots q_0)_b, \quad r = (r_{n-1}\ldots r_1 r_0)_b
\] 
}

\frame[plain]{
  \frametitle{Алгоритм длинного деления}
  \begin{codebox}
    \li $d \gets \lfloor b/(v_{n-1}+1)\rfloor$, $u \gets u \times d$,
    $v \gets v \times d$. 
    \zi \Comment В $u$ может появиться дополнительный разряд (до $n+m$).
    \li \For $j \gets m$ \Downto $0$ \Do
    \li $\hat{q} \gets \lfloor (u_{j+n}b + u_{j+n-1})/v_{n-1}\rfloor$
    \li $\hat{r} \gets (u_{j+n}b + u_{j+n-1}) \mod v_{n-1}$
    \li \While $\hat{r}<b \quad \func{and} \quad (\hat{q} \isequal b
    \quad \func{or} \quad \hat{q}v_{n-2} > b\hat{r}+u_{j+n-2}$) \Do
    \li $\hat{q} \gets \hat{q}-1$, $\hat{r} \gets \hat{r}+v_{n-1}$
    \End
    \li $(u_{j+n}\ldots u_j)_b \gets (u_{j+n}\ldots u_j)_b -
    \hat{q}(v_{n-1}\ldots v_0)_b$
    \li $q_j \gets q$
    \li \If $(u_{j+n}\ldots u_j)_b < 0$ \Then 
    \li $q_j \gets q_j -1$
    \li $(u_{j+n}\ldots u_j)_b \gets (u_{j+n}\ldots u_j)_b+(0
    v_{n-1}\ldots v_0)_b$
    \zi \Comment Перенос в разряд $u_{j+n-1}$ игнорируется.
    \End
    \End
    \li $r \gets (u_{n-1}\ldots u_0)_b/d$
  \end{codebox}
}

\section{Быстрое умножение}

\subsection{Алгоритм Карацубы}

\frame{
  \frametitle{Можно ли быстрее, чем за $n^2$?}
  
  Допустим, есть 
  \[u = (u_{2n-1}\ldots u_1u_0)_2=(U_1,U_0)_{2^n}, \quad v=(v_{2n-1}\ldots v_1
  v_0)_2=(V_1,V_0)_{2^n}\]
  Можно переписать:
  \[
  u = 2^nU_1 + U_0, \quad v = 2^nV_1+V_0
  \]
  Тогда
  \[
  uv = (2^{2n}+2^n)U_1V_1 + 2^n(U_1-U_0)(V_0-V_1)+(2^n+1)U_0V_0.
  \]
\pause
  Т.е., операция умножения $2n$-битных чисел свелась к 3 операциям
  умножения $n$-битных чисел и нескольких операций сдвига и сложения.

}

\frame{
  \frametitle{Быстрое умножение}
  Этот подход можно использовать для рекурсивного процесса
  умножения. Допустим, $T(n)$ --- время, затрачиваемое на выполнение
  умножения $n$-битных чисел. Тогда:
  \begin{align*}
  T(2n) &\leq 3T(n)+cn \\
  T(2^k)&\leq c(3^k-2^k), \qquad k \geq 1
  \end{align*}

  То есть:
\[
T(n) \leq T(2^{\lceil \lg n \rceil}) \leq c(3^{\lceil \lg n \rceil} -
2^{\lceil \lg n \rceil}) < 3c \cdot 3^{\lg n} = 3cn^{\lg 3}
\]
\pause
Можно сократить операцию умножения до $n^{1{.}585}$
}

\end{document}
    
